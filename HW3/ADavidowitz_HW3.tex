\documentclass{article}\usepackage[]{graphicx}\usepackage[]{xcolor}
% maxwidth is the original width if it is less than linewidth
% otherwise use linewidth (to make sure the graphics do not exceed the margin)
\makeatletter
\def\maxwidth{ %
  \ifdim\Gin@nat@width>\linewidth
    \linewidth
  \else
    \Gin@nat@width
  \fi
}
\makeatother

\definecolor{fgcolor}{rgb}{0.345, 0.345, 0.345}
\newcommand{\hlnum}[1]{\textcolor[rgb]{0.686,0.059,0.569}{#1}}%
\newcommand{\hlstr}[1]{\textcolor[rgb]{0.192,0.494,0.8}{#1}}%
\newcommand{\hlcom}[1]{\textcolor[rgb]{0.678,0.584,0.686}{\textit{#1}}}%
\newcommand{\hlopt}[1]{\textcolor[rgb]{0,0,0}{#1}}%
\newcommand{\hlstd}[1]{\textcolor[rgb]{0.345,0.345,0.345}{#1}}%
\newcommand{\hlkwa}[1]{\textcolor[rgb]{0.161,0.373,0.58}{\textbf{#1}}}%
\newcommand{\hlkwb}[1]{\textcolor[rgb]{0.69,0.353,0.396}{#1}}%
\newcommand{\hlkwc}[1]{\textcolor[rgb]{0.333,0.667,0.333}{#1}}%
\newcommand{\hlkwd}[1]{\textcolor[rgb]{0.737,0.353,0.396}{\textbf{#1}}}%
\let\hlipl\hlkwb

\usepackage{framed}
\makeatletter
\newenvironment{kframe}{%
 \def\at@end@of@kframe{}%
 \ifinner\ifhmode%
  \def\at@end@of@kframe{\end{minipage}}%
  \begin{minipage}{\columnwidth}%
 \fi\fi%
 \def\FrameCommand##1{\hskip\@totalleftmargin \hskip-\fboxsep
 \colorbox{shadecolor}{##1}\hskip-\fboxsep
     % There is no \\@totalrightmargin, so:
     \hskip-\linewidth \hskip-\@totalleftmargin \hskip\columnwidth}%
 \MakeFramed {\advance\hsize-\width
   \@totalleftmargin\z@ \linewidth\hsize
   \@setminipage}}%
 {\par\unskip\endMakeFramed%
 \at@end@of@kframe}
\makeatother

\definecolor{shadecolor}{rgb}{.97, .97, .97}
\definecolor{messagecolor}{rgb}{0, 0, 0}
\definecolor{warningcolor}{rgb}{1, 0, 1}
\definecolor{errorcolor}{rgb}{1, 0, 0}
\newenvironment{knitrout}{}{} % an empty environment to be redefined in TeX

\usepackage{alltt}
\title {DATA 605 HW Assignment 3}
\author{Avery Davidowitz}
\date{\today}

\usepackage[margin=40pt]{geometry}
\IfFileExists{upquote.sty}{\usepackage{upquote}}{}
\begin{document}
\maketitle
\bigbreak

\section{Problem Set 1}
\subsection{What is the rank of the matrix A?}

If:
\[ A =
\left[ 
\begin{array}{cccc}
1 & 2 & 3 & 4\\
-1 & 0 & 1 & 3\\
0 & 1 & -2 & 1\\
5 & 4 & -2 & -3
\end{array} 
\right] 
\]
Then the reduced row echelon form of A is:
\begin{knitrout}
\definecolor{shadecolor}{rgb}{0.969, 0.969, 0.969}\color{fgcolor}\begin{kframe}
\begin{alltt}
\hlkwd{library}\hlstd{(pracma)}
\hlstd{A} \hlkwb{=} \hlkwd{matrix}\hlstd{(}\hlkwd{c}\hlstd{(}\hlnum{1}\hlstd{,}\hlnum{2}\hlstd{,}\hlnum{3}\hlstd{,}\hlnum{4}\hlstd{,}
             \hlopt{-}\hlnum{1}\hlstd{,}\hlnum{0}\hlstd{,}\hlnum{1}\hlstd{,}\hlnum{3}\hlstd{,}
             \hlnum{0}\hlstd{,}\hlnum{1}\hlstd{,}\hlopt{-}\hlnum{2}\hlstd{,}\hlnum{1}\hlstd{,}
             \hlnum{5}\hlstd{,}\hlnum{4}\hlstd{,}\hlopt{-}\hlnum{2}\hlstd{,}\hlopt{-}\hlnum{3}\hlstd{),}
             \hlnum{4}\hlstd{,} \hlkwc{byrow}\hlstd{=}\hlnum{TRUE}\hlstd{)}
\hlkwd{rref}\hlstd{(A)}
\end{alltt}
\begin{verbatim}
##      [,1] [,2] [,3] [,4]
## [1,]    1    0    0    0
## [2,]    0    1    0    0
## [3,]    0    0    1    0
## [4,]    0    0    0    1
\end{verbatim}
\end{kframe}
\end{knitrout}
The reduced row echelon form of A being equal to $I_{n}$ is equivalent to A having rank n.
Therefore rank(A) = 4

\subsection{Given an $m \times n$ matrix where $m > n$, what can be the maximum rank? The mini-mum rank, assuming that the matrix is non-zero?}
Since the rank(A) is equal to the common dim of the column and row spaces, the maximum rank can be at most n. Assuming the matrix is non-zero, the minimum rank(A) is 1.
\subsection{What is the rank of matrix B?}
If:
\[ B =
\left[ 
\begin{array}{ccc}
1 & 2 & 1\\
3 & 6 & 3\\
2 & 4 & 2
\end{array} 
\right] 
\]
Then the reduced row echelon form of B is:
\begin{knitrout}
\definecolor{shadecolor}{rgb}{0.969, 0.969, 0.969}\color{fgcolor}\begin{kframe}
\begin{alltt}
\hlkwd{library}\hlstd{(pracma)}
\hlstd{B} \hlkwb{=} \hlkwd{matrix}\hlstd{(}\hlkwd{c}\hlstd{(}\hlnum{1}\hlstd{,}\hlnum{2}\hlstd{,}\hlnum{1}\hlstd{,}
             \hlnum{3}\hlstd{,}\hlnum{6}\hlstd{,}\hlnum{3}\hlstd{,}
             \hlnum{2}\hlstd{,}\hlnum{4}\hlstd{,}\hlnum{2}\hlstd{),}
             \hlnum{3}\hlstd{,} \hlkwc{byrow}\hlstd{=}\hlnum{TRUE}\hlstd{)}
\hlkwd{rref}\hlstd{(B)}
\end{alltt}
\begin{verbatim}
##      [,1] [,2] [,3]
## [1,]    1    2    1
## [2,]    0    0    0
## [3,]    0    0    0
\end{verbatim}
\end{kframe}
\end{knitrout}
The vectors with leading 1s form the basis for their respective row and column spaces of B, since B is in reduced row echelon form. This implies the row space of B has dim(1) and the column space has dim(3). Therefore, rank(B) = 1 due to the common dim of both spaces. Upon inspection, it is apparent that all rows of B are linearly dependent. R2 = 3 * R1 and R3 = 2 * R1.


\section{Problem Set 2}
Compute the eigenvalues and eigenvectors of the matrix A.

\[ A =
\left[ 
\begin{array}{ccc}
1 & 2 & 3\\
0 & 4 & 5\\
0 & 0 & 6
\end{array} 
\right] 
\]

\noindent The eigenvalues of matrix A are computed by solving the characteristic equation $\det(\lambda I - A) = 0$.
Since A is a triangular matrix the determinate is equal to the product of the diagonal.
\[\det(\lambda I - A) = (\lambda - 1)(\lambda - 4)(\lambda - 6) \]
Setting each component of the factored polynomial equal to 0 yields our eigenvalues of $\lambda = 1, 4, 6$ 
\bigbreak
\noindent The eigenvectors of A are the vectors $\vec{x}$ that satisfy $A\vec{x} = \lambda\vec{x}$ or alternatively satisfy $(\lambda I - A)\vec{x} = 0$
\\For $\lambda = 1$:

\[ (1 I - A) =
\left[ 
\begin{array}{ccc}
0 & -2 & -3\\
0 & -3 & -5\\
0 & 0 & -5
\end{array} 
\right] 
\]
\noindent The null space of a matrix is equal to the null space of the reduced row echelon form of that matrix. Applying the following row operations yields our RREF. R3 = R3 * -1/5, R2 = R2 + 5R3,  R2 = R2 * -1/3, R1 = R1 -2R2 - 3R3.
\[RREF(1I - A) = 
\left[
\begin{array}{ccc}
0 & 0 & 0\\
0 & 1 & 0\\
0 & 0 & 1
\end{array} 
\right] 
\]
Solving $(\lambda I - A)\vec{x} = 0$ $\rightarrow 0x_{1} = 0$, $x_{2} = 0$ and $x_{3} = 0$ yields the eigenspace for $\lambda = 1$ :
\[ E_{1} = \mbox{Span}
\left(
s\times
\left[
\begin{array}{c}
1\\
0\\
0
\end{array}
\right]
\forall s \in \mathbf{R}
\right)
\]

\noindent For $\lambda = 4$ :
 
\[ (4I - A) =
\left[
\begin{array}{ccc}
3 & -2 & -3\\
0 & 0 & -5\\
0 & 0 & -2
\end{array}
\right]
\]

\noindent Applying the following row operations yields our RREF. R3 = R3 * -1/2, R2 = R2 + 5R3, R1 = R1 + 3R3, R1 = R1 * 1/3.
\[RREF(4I - A) =
\left[
\begin{array}{ccc}
1 & -2/3 & 0\\
0 & 0 & 0\\
0 & 0 & 1
\end{array}
\right]
\]

\noindent Solving $(\lambda I - A)\vec{x} = 0$ $\rightarrow x_{3} = 0$ and $x_{1} = 2/3 x_{2}$. Setting $x_{2} =s$ and $x_{1} = 2/3s$ yields the eigenspace for $\lambda = 4$:

\[ E_{4} = \mbox{Span}
\left(
s\times
\left[
\begin{array}{c}
2/3\\
1\\
0
\end{array}
\right]
\forall s \in \mathbf{R}
\right)
\]

\noindent For $\lambda = 6$ :
 
\[ (6I - A) =
\left[
\begin{array}{ccc}
5 & -2 & -3\\
0 & 2 & -5\\
0 & 0 & 0
\end{array}
\right]
\]

\noindent Applying the following row operations yields our RREF. R2 = R2 * 1/2, R1 = R1 + 2R2, R1 = R1 * 1/5.
\[RREF(6I - A) =
\left[
\begin{array}{ccc}
1 & 0 & -8/5\\
0 & 1 & -5/2\\
0 & 0 & 0
\end{array}
\right]
\]

\noindent Setting $x_{3} =s$ and solving $(\lambda I - A)\vec{x} = 0$ $\rightarrow x_{1} = 8/5$s and $x_{2} = 5/2$s. This yields the eigenspace for $\lambda = 6$:

\[ E_{6} = \mbox{Span}
\left(
s\times
\left[
\begin{array}{c}
8/5\\
5/2\\
1
\end{array}
\right]
\forall s \in \mathbf{R}
\right)
\]

\noindent Verifying solutions with R
\begin{knitrout}
\definecolor{shadecolor}{rgb}{0.969, 0.969, 0.969}\color{fgcolor}\begin{kframe}
\begin{alltt}
\hlstd{A} \hlkwb{=} \hlkwd{matrix}\hlstd{(}\hlkwd{c}\hlstd{(}\hlnum{1}\hlstd{,}\hlnum{2}\hlstd{,}\hlnum{3}\hlstd{,}
             \hlnum{0}\hlstd{,}\hlnum{4}\hlstd{,}\hlnum{5}\hlstd{,}
             \hlnum{0}\hlstd{,}\hlnum{0}\hlstd{,}\hlnum{6}\hlstd{),}
             \hlnum{3}\hlstd{,} \hlkwc{byrow}\hlstd{=}\hlnum{TRUE}\hlstd{)}
\hlstd{ev} \hlkwb{<-} \hlkwd{eigen}\hlstd{(A)}
\hlstd{values} \hlkwb{<-} \hlstd{ev}\hlopt{$}\hlstd{values}
\hlkwd{print}\hlstd{(values)}
\end{alltt}
\begin{verbatim}
## [1] 6 4 1
\end{verbatim}
\begin{alltt}
\hlstd{vectors} \hlkwb{<-} \hlstd{ev}\hlopt{$}\hlstd{vectors}
\hlkwd{print}\hlstd{(vectors)}
\end{alltt}
\begin{verbatim}
##           [,1]      [,2] [,3]
## [1,] 0.5108407 0.5547002    1
## [2,] 0.7981886 0.8320503    0
## [3,] 0.3192754 0.0000000    0
\end{verbatim}
\end{kframe}
\end{knitrout}

\end{document}
